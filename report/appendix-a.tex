\phantomsection\section*{ПРИЛОЖЕНИЕ А}\addcontentsline{toc}{section}{ПРИЛОЖЕНИЕ А}

% Презентация к научно-исследовательской работе содержит 7 слайдов.

% \includepdf[pages={1-7},angle=90,scale=0.8,pagecommand={\thispagestyle{plain}}]{presentation.pdf}

\begin{lstlisting}[caption={Функция для выделения характеристик строки пикселей}, label={lst:elf}]
def extract_line_features(
    scanline: np.ndarray,
    left_margin: int = 0,
    right_margin: Optional[int] = None,
    white_thresh: int = WHITE_THRESH,
    gray_tol: int = GRAY_TOL,
) -> LineFeatures:
    end = right_margin if right_margin is not None else scanline.shape[0]
    segment = scanline[left_margin:end]
    if segment.size == 0:
        raise ValueError("Segment length is zero. Проверьте left_margin/right_margin.")

    (mask_white,
     mask_nonwhite,
     mask_color,
     mask_gray) = get_masks(segment, white_thresh, gray_tol)

    # Счётчики
    count_white = int(np.sum(mask_white))
    count_color = int(np.sum(mask_color))
    count_gray = int(np.sum(mask_gray))

    def rle_lengths(mask: np.ndarray) -> List[int]:
        if mask.ndim != 1:
            mask = mask.ravel()
        padded = np.pad(mask.astype(np.int8), (1, 1), mode='constant')
        diffs = np.diff(padded)
        run_starts = np.where(diffs == 1)[0]
        run_ends = np.where(diffs == -1)[0]
        return (run_ends - run_starts).tolist()

    comp_lengths = rle_lengths(mask_nonwhite)
    gray_comp_lengths = rle_lengths(mask_gray)
    color_comp_lengths = rle_lengths(mask_color)

    # gap_lengths: разности между концами и началом следующих компонент
    padded_nonwhite = np.pad(mask_nonwhite.astype(np.uint8), (1, 1))
    diffs = np.diff(padded_nonwhite)
    starts = np.where(diffs == 1)[0]
    ends = np.where(diffs == -1)[0]

    # Защита от несогласованных размеров: отбрасываем лишние значения
    min_len = min(len(starts), len(ends))
    starts = starts[:min_len]
    ends = ends[:min_len]

    # Если после обрезки осталось хотя бы 2 компоненты, можно искать промежутки
    if len(starts) > 1:
        gap_lengths = (starts[1:] - ends[:-1]).tolist()
    else:
        gap_lengths = []

    # Первый не-белый пиксель
    nonwhite_idxs = np.flatnonzero(mask_nonwhite)
    first_nonwhite_index = int(nonwhite_idxs[0] + left_margin) if nonwhite_idxs.size > 0 else None

    return LineFeatures(
        count_white,
        count_color,
        count_gray,
        comp_lengths,
        gap_lengths,
        gray_comp_lengths,
        color_comp_lengths,
        first_nonwhite_index,
    )
\end{lstlisting}

\begin{lstlisting}[caption={Функция классификации строки}, label={lst:classify_line}]
def classify_line(feat: LineFeatures):
    cond_background = (
        feat.first_nonwhite_index is None
    )
    if cond_background:
        return State.BACKGROUND

    min_long_black_line_length = get_min_long_black_line_length(feat)
    has_single_gray_comp = (
        len(feat.comp_lengths) == 1 and
        len(feat.gap_lengths) == 0 and
        len(feat.color_comp_lengths) == 0
    )
    pretty_long_gray_comp = (
        feat.count_gray > min_long_black_line_length
    )
    cond_long_black_line = (
        has_single_gray_comp and
        pretty_long_gray_comp
    )
    if cond_long_black_line:
        return State.LONG_BLACK_LINE

    min_medium_black_line_length = get_min_medium_black_line_length(feat)
    has_medium_sized_gray_comp = (
        any(i > min_medium_black_line_length
            for i in feat.gray_comp_lengths)
    )
    cond_medium_black_line = (
        has_medium_sized_gray_comp
    )
    if cond_medium_black_line:
        return State.MEDIUM_BLACK_LINE

    n = MIN_NUMBER_OF_COMPONENTS_TO_BE_CONSIDERED_A_LOT
    has_a_fucking_lot_of_comps = (
        len(feat.comp_lengths) > MIN_NUMBER_OF_COMPONENTS_TO_BE_CONSIDERED_A_FUCKNIG_LOT
    )
    has_a_lot_of_comps_and_no_color = (
        len(feat.comp_lengths) > n and
        feat.count_color == 0
    )
    cond_many_text = (
        has_a_fucking_lot_of_comps or
        has_a_lot_of_comps_and_no_color
    )
    if cond_many_text:
        return State.MANY_TEXT

    has_color = (
        feat.count_color > 0
    )
    cond_color = (
        has_color
    )
    if cond_color:
        return State.COLOR

    if len(feat.comp_lengths) == 0:
        mean_comp = 0
    else:
        mean_comp = np.mean(feat.comp_lengths)

    if len(feat.gap_lengths) == 0:
        mean_gap = 0
        std_gap = 0
        z_scores = []
    else:
        mean_gap = np.mean(feat.gap_lengths)
        std_gap = np.std(feat.gap_lengths)
        z_scores = (np.array(feat.gap_lengths) - mean_gap) / std_gap

    has_huge_gaps = any(abs(z) > HUGE_GAP_ZSCORE for z in z_scores)
    has_a_few_comps = (
        len(feat.comp_lengths) <= n
    )
    comps_are_mostly_small = (
        mean_comp < MAX_COMPONENT_LENGTH_TO_BE_CONSIDERED_SMALL
    )
    gaps_are_mostly_small = (
        mean_gap < MAX_COMPONENT_LENGTH_TO_BE_CONSIDERED_SMALL
    )
    cond_few_text = (
        has_a_few_comps and
        comps_are_mostly_small and
        gaps_are_mostly_small and
        not has_huge_gaps
    )
    if cond_few_text:
        return State.FEW_TEXT

    return State.UNDEFINED
\end{lstlisting}

\begin{lstlisting}[caption={Функция для создания первичной или уточненной разметки}, label={lsf:sd}]
def segment_document(
    image: np.ndarray,
    line_feature_func: Callable[[np.ndarray], LineFeatures],
    raw: bool = False,
):
    empty_line = np.zeros_like(image[0:1]).reshape(-1, image[0:1].shape[-1]).min(axis=-1).astype(int)
    def empty_segment_data():
        return SegmentData(
            0, 0, 0, 0, 0, 0, 0, 0, 0, 0, 0, 0, 0, 0,
            empty_line.copy(), empty_line.copy()
        )

    def reset_segment_data(sd: SegmentData):
        sd.start = sd.end
        sd.count_single_long_black_line = 0
        sd.count_single_medium_black_line = 0
        sd.count_long_black_line = 0
        sd.count_medium_black_line = 0
        sd.count_total_medium_black_line = 0
        sd.count_many_text = 0
        sd.count_color = 0
        sd.count_few_text = 0
        sd.count_undefined = 0
        sd.count_white_px = 0
        sd.count_color_px = 0
        sd.count_gray_px = 0
        sd.heatmap_black = np.zeros_like(empty_line)
        sd.heatmap_color = np.zeros_like(empty_line)

    results = []
    height = image.shape[0]
    prev_state = State.BACKGROUND
    prev_feat = line_feature_func(image[0:1])
    sd = empty_segment_data()
    for y in range(1, height):
        line = image[y:y+1]
        feat = line_feature_func(line)
        state = update_state(prev_state, feat)

        bg_started = state == State.BACKGROUND and prev_state != State.BACKGROUND
        bg_finished = state != State.BACKGROUND and prev_state == State.BACKGROUND
        if bg_started or bg_finished:
            class_name = classify_segment(prev_state, sd, raw)
            result = (sd.start, sd.end, class_name)
            results.append(result)
            reset_segment_data(sd)

        update_segment_data(sd, prev_feat, feat, line)
        prev_state = state
        prev_feat = feat
    class_name = classify_segment(prev_state, sd, raw)

    result = (sd.start, sd.end, class_name)
    results.append(result)

    return results
\end{lstlisting}

\begin{lstlisting}[caption={Конечный автомат}, label={lst:fsm}]
FSM = {
    State.BACKGROUND: {
        State.UNDEFINED:         State.UNDEFINED,
        State.MANY_TEXT:         State.MANY_TEXT,
        State.FEW_TEXT:          State.FEW_TEXT,
        State.LONG_BLACK_LINE:   State.LONG_BLACK_LINE,
        State.MEDIUM_BLACK_LINE: State.MEDIUM_BLACK_LINE,
        State.COLOR:             State.COLOR,
        State.BACKGROUND:        State.BACKGROUND,
    },
    State.UNDEFINED: {
        State.UNDEFINED:         State.UNDEFINED,
        State.MANY_TEXT:         State.MANY_TEXT,
        State.FEW_TEXT:          State.UNDEFINED,
        State.LONG_BLACK_LINE:   State.UNDEFINED,
        State.MEDIUM_BLACK_LINE: State.MEDIUM_BLACK_LINE,
        State.COLOR:             State.COLOR,
        State.BACKGROUND:        State.BACKGROUND,
    },
    State.MANY_TEXT: {
        State.UNDEFINED:         State.MANY_TEXT,
        State.MANY_TEXT:         State.MANY_TEXT,
        State.FEW_TEXT:          State.MANY_TEXT,
        State.LONG_BLACK_LINE:   State.MANY_TEXT,
        State.MEDIUM_BLACK_LINE: State.MEDIUM_BLACK_LINE,
        State.COLOR:             State.MANY_TEXT,
        State.BACKGROUND:        State.BACKGROUND,
    },
    State.FEW_TEXT: {
        State.UNDEFINED:         State.UNDEFINED,
        State.MANY_TEXT:         State.MANY_TEXT,
        State.FEW_TEXT:          State.FEW_TEXT,
        State.LONG_BLACK_LINE:   State.LONG_BLACK_LINE,
        State.MEDIUM_BLACK_LINE: State.MEDIUM_BLACK_LINE,
        State.COLOR:             State.COLOR,
        State.BACKGROUND:        State.BACKGROUND,
    },
    State.LONG_BLACK_LINE: {
        State.UNDEFINED:         State.LONG_BLACK_LINE,
        State.MANY_TEXT:         State.LONG_BLACK_LINE,
        State.FEW_TEXT:          State.LONG_BLACK_LINE,
        State.LONG_BLACK_LINE:   State.LONG_BLACK_LINE,
        State.MEDIUM_BLACK_LINE: State.MEDIUM_BLACK_LINE,
        State.COLOR:             State.LONG_BLACK_LINE,
        State.BACKGROUND:        State.BACKGROUND,
    },
    State.MEDIUM_BLACK_LINE: {
        State.UNDEFINED:         State.MEDIUM_BLACK_LINE,
        State.MANY_TEXT:         State.MEDIUM_BLACK_LINE,
        State.FEW_TEXT:          State.MEDIUM_BLACK_LINE,
        State.LONG_BLACK_LINE:   State.LONG_BLACK_LINE,
        State.MEDIUM_BLACK_LINE: State.MEDIUM_BLACK_LINE,
        State.COLOR:             State.COLOR,
        State.BACKGROUND:        State.BACKGROUND,
    },
    State.COLOR: {
        State.UNDEFINED:         State.COLOR,
        State.MANY_TEXT:         State.COLOR,
        State.FEW_TEXT:          State.COLOR,
        State.LONG_BLACK_LINE:   State.LONG_BLACK_LINE,
        State.MEDIUM_BLACK_LINE: State.MEDIUM_BLACK_LINE,
        State.COLOR:             State.COLOR,
        State.BACKGROUND:        State.BACKGROUND,
    },
}
\end{lstlisting}

\begin{lstlisting}[caption={Структура State}, label={lst:state}]
class State:
    UNDEFINED         = 0
    BACKGROUND        = 1
    FEW_TEXT          = 2
    MANY_TEXT         = 3
    COLOR             = 4
    MEDIUM_BLACK_LINE = 5
    LONG_BLACK_LINE   = 6
\end{lstlisting}

\begin{lstlisting}[caption={Функция handle\_undefined}, label={lst:hundef}]
def handle_undefined(sd: SegmentData):
    def text(sd: SegmentData):
        height = sd.end - sd.start
        not_very_high = height < MAX_SEGMENT_HEIGHT_TO_BE_CONSIDERED_NOT_VERY_HIGH
        had_a_lot_of_few_text = (sd.count_few_text / height) > MIN_FEW_TEXT_RATIO_TO_BE_CONSIDERED_A_LOT
        return (
            not_very_high or
            had_a_lot_of_few_text
        )

    def code(sd: SegmentData):
        height = sd.end - sd.start

        high_vbls = sd.heatmap_black == height
        padded = np.concatenate(([False], high_vbls, [False]))
        diff = np.diff(padded.astype(int))
        n_vertical_black_lines = len(np.where(diff == 1)[0])

        return n_vertical_black_lines == 2

    def figure(sd: SegmentData):
        height = sd.end - sd.start
        pretty_high = height > MIN_FIGURE_HEIGHT_TO_BE_CONSIDERED_HIGH
        return pretty_high

    def plot(sd: SegmentData):
        height = sd.end - sd.start

        high_vbls = sd.heatmap_black >= PLOT_VERTICAL_LINE_HEIGHT_CORRECTION * height
        padded = np.concatenate(([False], high_vbls, [False]))
        diff = np.diff(padded.astype(int))
        n_vertical_black_lines = len(np.where(diff == 1)[0])

        return n_vertical_black_lines == 1

    if text(sd):
        return ClassNames[Class.TEXT]

    if code(sd):
        return ClassNames[Class.CODE]

    if figure(sd):
        return ClassNames[Class.FIGURE]

    if plot(sd):
        return ClassNames[Class.PLOT]

    return ClassNames[Class.UNDEFINED]
\end{lstlisting}

\begin{lstlisting}[caption={Функция handle\_background}, label={lst:}]
def handle_background(_: SegmentData):
    return ClassNames[Class.BACKGROUND]
\end{lstlisting}

\begin{lstlisting}[caption={Функция handle\_few\_text}, label={lst:}]
def handle_few_text(_: SegmentData):
    return ClassNames[Class.TEXT]
\end{lstlisting}

\begin{lstlisting}[caption={Функция handle\_many\_text}, label={lst:}]
def handle_many_text(sd: SegmentData):
    def table(sd: SegmentData):
        height = sd.end - sd.start

        high_vbls = sd.heatmap_black == height
        padded = np.concatenate(([False], high_vbls, [False]))
        diff = np.diff(padded.astype(int))
        lbl_start_indices = np.where(diff == 1)[0]
        n_vertical_black_lines = len(lbl_start_indices)
        has_more_than_two_vertical_lines = n_vertical_black_lines > 2

        min_space_is_reasonably_small = True
        if has_more_than_two_vertical_lines:
            min_space_is_reasonably_small = min(np.diff(lbl_start_indices)) >
          MIN_REASONABLY_SMALL_SPACE_BETWEEN_TWO_COLUMNS_IN_TABLE


        pretty_high = height > MIN_SEGMENT_HEIGHT_TO_BE_CONSIDERED_HIGH

        return (
            pretty_high and
            has_more_than_two_vertical_lines and
            min_space_is_reasonably_small
        )

    def code(sd: SegmentData):
        height = sd.end - sd.start

        high_vbls = sd.heatmap_black == height
        padded = np.concatenate(([False], high_vbls, [False]))
        diff = np.diff(padded.astype(int))
        n_vertical_black_lines = len(np.where(diff == 1)[0])

        has_two_vertical_lines = n_vertical_black_lines == 2
        has_no_mbls = sd.count_single_medium_black_line == 0
        had_many_text = sd.count_many_text > 0
        has_no_color = sd.count_color == 0

        pretty_high = height > MIN_SEGMENT_HEIGHT_TO_BE_CONSIDERED_HIGH

        return (
            pretty_high and
            has_two_vertical_lines and
            has_no_mbls and
            (
                had_many_text or
                has_no_color
            )
        )

    if table(sd):
        return ClassNames[Class.TABLE]

    if code(sd):
        return ClassNames[Class.CODE]

    return ClassNames[Class.TEXT]
\end{lstlisting}

\begin{lstlisting}[caption={Функция handle\_color}, label={lst:}]
def handle_color(sd: SegmentData):
    def plot(sd: SegmentData):
        height = sd.end - sd.start

        high_vbls = sd.heatmap_black >= PLOT_VERTICAL_LINE_HEIGHT_CORRECTION * height
        padded = np.concatenate(([False], high_vbls, [False]))
        diff = np.diff(padded.astype(int))
        n_vertical_black_lines = len(np.where(diff == 1)[0])

        has_single_vertical_axis = n_vertical_black_lines == 1
        has_pretty_small_color_to_white_relation = (sd.count_color_px /
            sd.count_white_px) < MIN_COLOR_TO_WHITE_RATIO_TO_BE_CONSIDERED_SMALL
        return (
            has_single_vertical_axis and
            has_pretty_small_color_to_white_relation
        )

    def undefined(sd: SegmentData):
        height = sd.end - sd.start
        smol = height < MAX_SEGMENT_HEIGHT_TO_BE_CONSIDERED_SMALL
        return smol

    if plot(sd):
        return ClassNames[Class.PLOT]

    if undefined(sd):
        return ClassNames[Class.UNDEFINED]
    
    return ClassNames[Class.FIGURE]
\end{lstlisting}

\begin{lstlisting}[caption={Функция handle\_medium\_black\_line}, label={lst:}]
def handle_medium_black_line(sd: SegmentData):
    def text(sd: SegmentData):
        height = sd.end - sd.start
        had_a_lot_of_a_lot_of_text = (sd.count_many_text / height) > MIN_MANY_TEXT_RATIO_TO_BE_CONSIDERED_A_LOT
        pretty_small = height <
        MAX_SEGMENT_HEIGHT_FOR_A_TEXT_TO_BE_CONSIDERED_NOT_SMALL
        had_a_lot_of_undefined = (sd.count_undefined / height) > MIN_UNDEFINED_RATIO_TO_BE_CONSIDERED_A_LOT
        had_a_lot_of_few_text = (sd.count_few_text / height) > MIN_FEW_TEXT_RATIO_TO_BE_CONSIDERED_A_LOT
        return (
            had_a_lot_of_a_lot_of_text or
            (
                pretty_small and
                (
                    had_a_lot_of_undefined or
                    had_a_lot_of_few_text
                )
            )
        )

    def figure(sd: SegmentData):
        height = sd.end - sd.start
        has_color = sd.count_color > 0
        has_a_lot_of_mbls = (sd.count_medium_black_line / height) > MAX_MBLS_RATIO_TO_BE_CONSIDERED_FEW
        pretty_high = height > MIN_SEGMENT_HEIGHT_TO_BE_CONSIDERED_HIGH
        return (
            pretty_high and
            (
                has_color or
                has_a_lot_of_mbls
            )
        )

    def plot(sd: SegmentData):
        height = sd.end - sd.start
        has_color = sd.count_color > 0
        has_a_few_mbls = (sd.count_medium_black_line / height) < MAX_MBLS_RATIO_TO_BE_CONSIDERED_FEW

        high_vbls = sd.heatmap_black >= PLOT_VERTICAL_LINE_HEIGHT_CORRECTION * height
        padded = np.concatenate(([False], high_vbls, [False]))
        diff = np.diff(padded.astype(int))
        n_vertical_black_lines = len(np.where(diff == 1)[0])

        all_px = sd.count_white_px + sd.count_color_px + sd.count_gray_px
        has_many_white_pixels = sd.count_white_px / all_px > MIN_WHITE_PIXELS_RATIO_TO_BE_CONSIDERED_MANY

        return (
            has_color and
            has_a_few_mbls and
            n_vertical_black_lines >= 2 and
            has_many_white_pixels
        )

    def equation(sd: SegmentData):
        height = sd.end - sd.start
        not_very_high = height < MIN_SEGMENT_HEIGHT_TO_BE_CONSIDERED_HIGH
        has_a_few_mbls = sd.count_single_medium_black_line < MAX_MBLS_TO_BE_CONSIDERED_FEW
        has_single_mbl = sd.count_single_medium_black_line == 1
        return (
            (
                not_very_high and
                has_a_few_mbls
            ) or
            has_single_mbl
        )

    def diagram(sd: SegmentData):
        return sd.count_single_medium_black_line > 1

    def undefined(sd: SegmentData):
        height = sd.end - sd.start
        smol = height < MAX_SEGMENT_HEIGHT_TO_BE_CONSIDERED_SMALL
        return smol

    if plot(sd):
        return ClassNames[Class.PLOT]

    if figure(sd):
        return ClassNames[Class.FIGURE]

    if diagram(sd):
        return ClassNames[Class.DIAGRAM]

    if text(sd):
        return ClassNames[Class.TEXT]

    if equation(sd):
        return ClassNames[Class.UNDEFINED]

    if undefined(sd):
        return ClassNames[Class.UNDEFINED]

    return ClassNames[Class.DIAGRAM]
\end{lstlisting}

\begin{lstlisting}[caption={Функция handle\_long\_black\_line}, label={lst:hlbl}]
def handle_long_black_line(sd: SegmentData):
    def table(sd: SegmentData):
        height = sd.end - sd.start

        high_vbls = sd.heatmap_black == height
        padded = np.concatenate(([False], high_vbls, [False]))
        diff = np.diff(padded.astype(int))
        lbl_start_indices = np.where(diff == 1)[0]
        n_vertical_black_lines = len(lbl_start_indices)
        has_more_than_two_vertical_lines = n_vertical_black_lines > 2

        min_space_is_reasonably_small = True
        if has_more_than_two_vertical_lines:
            min_space_is_reasonably_small = min(np.diff(lbl_start_indices)) >
          MIN_REASONABLY_SMALL_SPACE_BETWEEN_TWO_COLUMNS_IN_TABLE

        return (
            has_more_than_two_vertical_lines and
            min_space_is_reasonably_small
        )

    def code(sd: SegmentData):
        height = sd.end - sd.start

        high_vbls = sd.heatmap_black == height
        padded = np.concatenate(([False], high_vbls, [False]))
        diff = np.diff(padded.astype(int))
        n_vertical_black_lines = len(np.where(diff == 1)[0])

        has_two_vertical_lines = n_vertical_black_lines == 2
        has_no_mbls = sd.count_single_medium_black_line == 0
        had_many_text = sd.count_many_text > 0
        has_no_color = sd.count_color == 0

        return (
            has_two_vertical_lines and
            has_no_mbls and
            (
                had_many_text or
                has_no_color
            )
        )

    def diagram(sd: SegmentData):
        has_no_color = sd.count_color == 0
        has_a_few_mbls = sd.count_single_medium_black_line >= 2
        return (
            has_no_color and
            has_a_few_mbls
        )

    def plot(sd: SegmentData):
        height = sd.end - sd.start
        has_color = sd.count_color > 0
        has_a_few_lbls = (sd.count_long_black_line / height) < 0.1

        high_vbls = sd.heatmap_black >= PLOT_VERTICAL_LINE_HEIGHT_CORRECTION * height
        padded = np.concatenate(([False], high_vbls, [False]))
        diff = np.diff(padded.astype(int))
        n_vertical_black_lines = len(np.where(diff == 1)[0])

        return (
            has_color and
            has_a_few_lbls and
            n_vertical_black_lines >= 2
        )

    def undefined(sd: SegmentData):
        height = sd.end - sd.start
        smol = height < MAX_SEGMENT_HEIGHT_TO_BE_CONSIDERED_SMALL
        return smol

    if undefined(sd):
        return ClassNames[Class.UNDEFINED]

    if plot(sd):
        return ClassNames[Class.PLOT]

    if table(sd):
        return ClassNames[Class.TABLE]

    if code(sd):
        return ClassNames[Class.CODE]

    if diagram(sd):
        return ClassNames[Class.DIAGRAM]

    return ClassNames[Class.FIGURE]
\end{lstlisting}

\begin{lstlisting}[caption={Функция обновления информации о сегменте}, label={lst:usd}]
def update_segment_data(sd: SegmentData, prev_feat, feat: LineFeatures, line: np.ndarray):
    prev_state = classify_line(prev_feat)
    state = classify_line(feat)

    sd.end += 1

    if prev_state != state and state == State.LONG_BLACK_LINE:
        sd.count_single_long_black_line += 1
    elif state == State.LONG_BLACK_LINE:
        sd.count_long_black_line += 1
    elif prev_state != state and state == State.MEDIUM_BLACK_LINE:
        sd.count_single_medium_black_line += 1
    elif state == State.MEDIUM_BLACK_LINE:
        sd.count_medium_black_line += 1
    elif state == State.MANY_TEXT:
        sd.count_many_text += 1
    elif state == State.COLOR:
        sd.count_color += 1
    elif state == State.FEW_TEXT:
        sd.count_few_text += 1
    elif state == State.UNDEFINED:
        sd.count_undefined += 1
    elif state == State.BACKGROUND:
        return

    sd.count_white_px += feat.count_white
    sd.count_color_px += feat.count_color
    sd.count_gray_px += feat.count_gray

    min_medium_black_line_length = get_min_medium_black_line_length(feat)

    # NOTE: Account for several MBLs on single line
    count_total_medium_black_lines = sum(np.array(feat.gray_comp_lengths) >
                                   min_medium_black_line_length)
    sd.count_total_medium_black_line += count_total_medium_black_lines

    (_, _, mask_color, mask_gray) = get_masks(line, WHITE_THRESH, GRAY_TOL)

    sd.heatmap_black += mask_gray.astype(int)
    sd.heatmap_color += mask_color.astype(int)
\end{lstlisting}

\begin{lstlisting}[caption={Функция создания объединенной разметки}, label={lst:merged}]
def merge(markup: List[Tuple[int, int, str]]):
    tmp_markup = [markup[0]]
    new_markup = [markup[0]]
    prev_start = curr_start = next_start = prev_end = curr_end = next_end = 0
    prev_height = curr_height = next_height = 0
    prev_class = curr_class = next_class = "None"
    
    # Merge small Background segment with nearest larger segment
    for i in range(1, len(markup) - 1):
        (prev_start, prev_end, prev_class) = markup[i-1]
        (curr_start, curr_end, curr_class) = markup[i]
        (next_start, next_end, next_class) = markup[i+1]

        prev_height = prev_end - prev_start
        curr_height = curr_end - curr_start
        next_height = next_end - next_start

        if curr_class == ClassNames[Class.BACKGROUND] and curr_height <
      MAX_SEGMENT_HEIGHT_TO_BE_MERGED_WITH_NEAREST_LARGER_SEGMENT:
            if prev_height > curr_height:
                curr_class = prev_class
            elif next_height > curr_height:
                curr_class = next_class

        segment = (curr_start, curr_end, curr_class)
        tmp_markup.append(segment)

    segment = (next_start, next_end, next_class)
    tmp_markup.append(segment)

    new_markup = merge_segments(tmp_markup)
    tmp_markup = [new_markup[0]]

    # Remove background between same-class segments
    for i in range(1, len(new_markup) - 1):
        (prev_start, prev_end, prev_class) = new_markup[i-1]
        (curr_start, curr_end, curr_class) = new_markup[i]
        (next_start, next_end, next_class) = new_markup[i+1]

        prev_height = prev_end - prev_start
        curr_height = curr_end - curr_start
        next_height = next_end - next_start

        if (
            prev_class == next_class and
            curr_class == ClassNames[Class.BACKGROUND]
        ):
            curr_class = next_class

        segment = (curr_start, curr_end, curr_class)
        tmp_markup.append(segment)

    segment = (next_start, next_end, next_class)
    tmp_markup.append(segment)

    new_markup = merge_segments(tmp_markup)
    tmp_markup = [new_markup[0]]

    # Make small backgrounds uncertain
    for i in range(len(new_markup)):
        (curr_start, curr_end, curr_class) = new_markup[i]
        curr_height = curr_end - curr_start

        if curr_class == ClassNames[Class.BACKGROUND] and curr_height < MAX_BACKGROUND_HEIGHT_TO_BECOME_UNDEFINED:
            curr_class = ClassNames[Class.UNDEFINED]

        segment = (curr_start, curr_end, curr_class)
        tmp_markup.append(segment)

    new_markup = merge_segments(tmp_markup)
    tmp_markup = [new_markup[0]]

    # Merge uncertainty with greatest neighbor
    for i in range(1, len(new_markup) - 1):
        (prev_start, prev_end, prev_class) = new_markup[i-1]
        (curr_start, curr_end, curr_class) = new_markup[i]
        (next_start, next_end, next_class) = new_markup[i+1]

        prev_height = prev_end - prev_start
        curr_height = curr_end - curr_start
        next_height = next_end - next_start

        if curr_class == ClassNames[Class.UNDEFINED] and curr_height < MAX_UNDEFINED_HEIGHT_TO_BE_MERGED:
            if prev_class != ClassNames[Class.BACKGROUND] and prev_height > curr_height:
                curr_class = prev_class
            elif next_class != ClassNames[Class.BACKGROUND] and next_height > curr_height:
                curr_class = next_class

        segment = (curr_start, curr_end, curr_class)
        tmp_markup.append(segment)

    if next_class == ClassNames[Class.UNDEFINED] and next_height < MAX_UNDEFINED_HEIGHT_TO_BE_MERGED:
        next_class = curr_class

    segment = (next_start, next_end, next_class)
    tmp_markup.append(segment)

    tmp_markup = tmp_markup[1:]
    second_class = tmp_markup[1][2]
    first_start = tmp_markup[0][0]
    first_end = tmp_markup[0][1]
    tmp_markup[0] = [first_start, first_end, second_class]
    if tmp_markup[-1][2] == ClassNames[Class.BACKGROUND]:
        s = tmp_markup[-1][0]
        e = tmp_markup[-1][1]
        c = tmp_markup[-2][2]
        tmp_markup[-1] = (s, e, c)

    new_markup = merge_segments(tmp_markup)

    return new_markup
\end{lstlisting}

