\phantomsection\section*{РЕФЕРАТ}\addcontentsline{toc}{section}{РЕФЕРАТ}

% NOTE:
% Текст реферата должен отражать
% - объект исследованияи и разработки
% - цель и задачи работы
% - результаты работы

% Отчет \pageref{LastPage} с., \total{figure} рис., \total{table} табл., 11 источн., 1 прил.

% DOCUMENT LAYOUT ANALYSIS,
% НАУЧНО-ТЕХНИЧЕСКИЙ ТЕКСТ,
% CONNECTED COMPONENT ANALYSIS,
% PROJECTION PROFILE ANALYSIS,
% RLSA,
% МАШИННОЕ ОБУЧЕНИЕ,
% КЛАССИФИКАЦИЯ
%
% Цель работы --- классификация методов выделения составных частей научного текста.
%
% В данной работе был проведен анализ предметных областей научно-технических текстов и анализа структуры документов.
% Был проведен обзор существующих методов выделения составных частей научного текста.
% Были сформулированы критерии сравнения описанных методов и была проведена классификацию описанных методов по сформулированным критериям.
