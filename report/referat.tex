\phantomsection\section*{РЕФЕРАТ}\addcontentsline{toc}{section}{РЕФЕРАТ}

Расчетно-пояснительная записка \pageref{LastPage} с., \total{figure} рис., \total{table} табл., 21 источн., 3 прил.

DOCUMENT LAYOUT ANALYSIS,
НАУЧНО-ТЕХНИЧЕСКИЙ ТЕК-СТ,
RLSA,
CONNECTED COMPONENT ANALYSIS,
PROJECTION PRO-FILE ANALYSIS,
МАШИННОЕ ОБУЧЕНИЕ,
КЛАССИФИКАЦИЯ,
РАЗМЕТКА,
ЭВРИСТИКА,
PYTHON

Объектом исследования являются методы, с помощью которых можно выделять составные части научного текста.

Объектом разработки является программное обеспечение, реализующее метод выделения составных частей научного текста на основе анализа распределения пикселей в сканирующей строке.

Целью работы является разработка метода выделения составных частей научного текста на основе анализа распределения пикселей в сканирующей строке.

В аналитическом разделе проведен анализ предметной области анализа структуры документов, приведена формализация задачи предметной области, описаны существующие методы и алгоритмы, применимые для задачи выделения составных частей научного текста, проведена классификация существующих методов, обоснована потребность в разработке нового метода, сформулирована цель данной работы и формализована постановка задачи.

В конструкторском разделе приведены требования и ограничения разрабатываемого метода, описаны основные этапы разрабатываемого метода, сценарии тестирования, классы эквивалентности тестов, а также структура разрабатываемого ПО.

В технологическом разделе выбраны средства реализации программного обеспечения, описаны основные функции разработанного программного обеспечения, приведены результаты тестирования, примеры пользовательского интерфейса, демонстрация работы программы, а также руководство пользователя.

В исследовательском разделе приведено описание исследования, технические характеристики устройства, на котором выполнялись замеры времени, представлены и проанализированы результаты исследования скорости разметки и максимального объема памяти, используемой в процессе разметки, в зависимости от количества процессов, участвующих в разметке.

Результаты проведенного исследования подтверждают начальные предположения о том, что разметка будет производиться наиболее эффективно в случае, когда количество рабочих процессов совпадает с количеством физических ядер процессора, а также тип разметки не влияет на максимальный объем используемой памяти.

% NOTE:
% Текст реферата должен отражать
% - объект исследованияи и разработки
% - цель и задачи работы
% - результаты работы


% DOCUMENT LAYOUT ANALYSIS,
% НАУЧНО-ТЕХНИЧЕСКИЙ ТЕКСТ,
% CONNECTED COMPONENT ANALYSIS,
% PROJECTION PROFILE ANALYSIS,
% RLSA,
% МАШИННОЕ ОБУЧЕНИЕ,
% КЛАССИФИКАЦИЯ
%
% Цель работы --- классификация методов выделения составных частей научного текста.
%
% В данной работе был проведен анализ предметных областей научно-технических текстов и анализа структуры документов.
% Был проведен обзор существующих методов выделения составных частей научного текста.
% Были сформулированы критерии сравнения описанных методов и была проведена классификацию описанных методов по сформулированным критериям.
