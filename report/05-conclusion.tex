\phantomsection\section*{ЗАКЛЮЧЕНИЕ}\addcontentsline{toc}{section}{ЗАКЛЮЧЕНИЕ}

В результате выполнения данной работы был разработан метод выделения составных частей научного текста на основе анализа распределения пикселей в сканирующей строке, для чего были решены следующие задачи:
\begin{enumerate}
    \item рассмотрены и сравнены известные методы, которые могут быть применены для выделения составных частей научного текста;
    \item формализована постановка задачи разработки метода выделения составных частей научного текста на основе анализа распределения пикселей в сканирующей строке;
    \item разработан описанный метод;
    \item разработано программное обеспечение, реализующее данный метод;
    \item проведено исследование скорости разметки и максимального объема памяти, используемой в процессе разметки, в зависимости от количества процессов, участвующих в разметке.
\end{enumerate}

В результате проведенного исследования начальные предположения, что разметка будет производиться наиболее эффективно в случае, когда количество рабочих процессов совпадает с количеством физических ядер процессора, а также тип разметки не влияет на максимальный объем используемой памяти, подтвердились.

% NOTE:
% Заключение содержит краткие выводы по всей работе и оценку полноты решения поставленной задачи

% В ходе данной научно-исследовательской работы был проведен анализ предметных областей научно-технических текстов и анализа структуры документов, проведен обзор существующих методов выделения составных частей научного текста, были сформулированы критерии сравнения описанных методов и была проведена классификацию описанных методов по сформулированным критериям.
%
% Таким образом, все задачи для достижения цели данной работы были решены, и цель работы --- классификация методов выделения составных частей научного текста --- была достигнута.
