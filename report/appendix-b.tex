\phantomsection\section*{ПРИЛОЖЕНИЕ Б}\addcontentsline{toc}{section}{ПРИЛОЖЕНИЕ Б}

Ниже описана структура приложений.

\noindent {src/apply.py } \dotfill {Скрипт для применения JSON разметки к PDF файлу}

\noindent {src/bench.py } \dotfill {Скрипт для запуска тестов производительности}

\noindent {src/debug.py } \dotfill {Отладочный скрипт с графическим веб-интерфейсом}

\noindent {src/fast.py } \dotfill {Файл, содержащий интерфейс для разметки}

\noindent {src/fsm.py } \dotfill {Файл, содержащий структуру конечного автомата}

\noindent {src/logic.py } \dotfill {Файл с основной логикой реализованного метода}

\noindent {src/main.py } \dotfill {Файл для запуска создания разметки}

\noindent {src/main-bench.py } \dotfill {Файл, используемый при замерах производительности}

\noindent {src/plot.py } \dotfill {Файл для построения графиков}

\noindent {src/pyproject.toml } \dotfill {Файл с зависимостями}

\noindent {src/states.py } \dotfill {Файл, описывающий классы для разметки}

\noindent {src/uv.lock } \dotfill {Файл блокировки зависимостей}

\noindent {src/webgui.py } \dotfill {Файл для запуска веб-интерфейса разметки}

\noindent {docs/vkr.pdf } \dotfill {Документ, содержащий данную РПЗ}

\noindent {docs/pres.pdf } \dotfill {Документ, содержащий презентацию к данной ВКР}
