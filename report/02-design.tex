\section{Конструкторский раздел}

% NOTE:
% В конструкторском разделе описывается разрабатываемый метод
% В случае если в бакалаврском проекте разрабатывается новый метод или алгоритм, необходимо подробно изложить их суть, привести все необходимые для их реализации математические выкладки, обосновать последовательность этапов выполнения
% При этом для каждого этапа следует выделить необходимые исходные данные и получаемые результаты
% Для описания метода или алгоритма - Схема алгоритма
% В конце описания разработанного алгоритма должны быть приведены выбранные способы тестирования и сами тесты
% Перед формированием тестовых наборов данных целесообразно указать выделенные классы эквивалентности
% (тут же могут быть приведены выкладки по теоретическим рассчетам требуемой памяти и эффективности алгоритма; эти результаты могут быть использованы для обоснования правильности выбора метода из уже имеющихся альтернативных вариантов)
% Также должно быть приведено описание структуры разрабатываемого ПО, оно включает в себя:
% - описание общей структуры - определение основных частей (компонентов) и их взаимосвязей по управлению и по данным
% - декомпозицию компонентов и построение структурных иерархий
% - проектирование компонентов
% Для графического представления такого описания (если есть необходимость), следует использовать IDEF0 с декомпозицией исходной задачи на несколько уровней

% Рек. Объем 25-30 страниц

\subsection{Требования и ограничения метода}

Метод выделения составных частей научного текста на основе анализа распределения пикселей в сканирующей строке должен:
\begin{enumerate}
    \item Работать с одноколоночными Манхэттенскими макетами документов;
    \item Выделять текстовые блоки;
    \item Выделять таблицы;
    \item Выделять листинги;
    \item Выделять схемы алгоритмов;
    \item Выделять рисунки;
    \item Выделять графики;
    \item Работать на основе простых правил и эвристик, без использования нейросетей.
\end{enumerate}

% Разработать метод выделения ... строке
% Изложить особенности предложенного метода

\subsection{Описание разрабатываемого метода}

Поставленная задача решается в четыре этапа:
\begin{enumerate}
    \item Преобразование PDF документа в изображения;
    \item Первичная разметка страниц;
    \item Создание уточненной разметки на основе первичной;
    \item Объединение уточненной разметку в более крупные блоки.
\end{enumerate}

Разметка, ее уточнение и объединение происходят на основе определенных правил, которые будут описаны в данном разделе далее.

Основные этапы разрабатываемого метода представлены на IDEF0 диаграмме первого уровня (см. Рисунок \ref{fig:a1}).

\begin{figure}[H]
	\centering
	\includegraphics[width=\textwidth]{diag/a1-big.pdf}
	\caption{IDEF0-диаграмма метода выделения составных частей научного текста на основе анализа распределения пикселей в сканирующей строке}
	\label{fig:a1}
\end{figure}

\subsubsection{Первичная разметка}

Исходные данные --- изображение страницы документа.
Получаемый результат --- разметка страницы и информация о каждом сегменте на ней.

Разметка страницы --- массив кортежей типа
$$
(y\_start, y\_end, C),
$$
где $y\_start$ --- $y$-координата начала сегмента в пространстве изображения, $y\_end$ --- $y$-координата конца сегмента в пространстве изображения, $C$ --- класс сегмента, где $C \subseteq$ $\{$Фон, Немного текста, Много текста, Цвет, Черная линия средней длины, Длинная черная линия, Не определено$\}$.

Информация о сегменте содержит следующие данные:
\begin{enumerate}
    \item start --- ордината начала сегмента;
    \item end --- ордината конца сегмента;
    \item count\_long\_black\_line --- количество раз, когда при разметке сегмента встретилась строка, идентифицированная, как <<Длинная черная линия>>;
    \item count\_single\_long\_black\_line --- количество раз, когда при разметке сегмента встретилась строка, идентифицированная, как <<Длинная черная линия>>, считая несколько подряд идущих <<Длинных черных линий>> за одну;
    \item count\_medium\_black\_line --- количество раз, когда при разметке сегмента встретилась строка, идентифицированная, как <<Черная линия средней длины>>;
    \item count\_single\_medium\_black\_line --- количество раз, когда при разметке сегмента встретилась строка, идентифицированная, как <<Черная линия средней длины>>, считая несколько подряд идущих <<Черных линий средней длины>> за одну;
    \item count\_total\_medium\_black\_line --- количество раз, когда при разметке сегмента встретилась строка, идентифицированная, как <<Черная линия средней длины>>, с учетом всех <<Черных линий черной длины>> если таких было зафиксировано несколько внутри одной сканирующей строки;
    \item count\_many\_text --- количество раз, когда при разметке сегмента встретилась строка, идентифицированная, как <<Много текста>>;
    \item count\_few\_text --- количество раз, когда при разметке сегмента встретилась строка, идентифицированная, как <<Немного текста>>;
    \item count\_color --- количество раз, когда при разметке сегмента встретилась строка, идентифицированная, как <<Цвет>>;
    \item count\_undefined --- количество раз, когда при разметке сегмента встретилась строка, идентифицированная, как <<Не определено>>;
    \item count\_white\_px --- количество белых пикселей в сегменте;
    \item count\_color\_px --- количество цветных пикселей в сегменте;
    \item count\_gray\_px --- количество черных пикселей в сегменте (сумма трех данных счетчиков дает общее количество пикселей в сегменте);
    \item heatmap\_black --- массив, $i$-й элемент которого отражает количество черных пикселей в $i$-й колонке пикселей сегмента;
    \item heatmap\_color --- массив, $i$-й элемент которого отражает количество цветных пикселей в $i$-й колонке пикселей сегмента.
\end{enumerate}

Данная информация будет использоваться для уточнения разметки в следующем этапе.

Первичная разметка создается в результате классификации строк на основе распределения пикселей в них и изменения состояний конечного автомата первичной разметки.

Диаграмма изменения состояний конечного автомата изображена на рисунке \ref{fig:fsm-full}.

\begin{figure}[H]
	\centering
	\includegraphics[width=\textwidth]{diag/fsm.full.pdf}
	\caption{Конечный автомат, все состояния}
	\label{fig:fsm-full}
\end{figure}

Из состояния <<Фон>> можно попасть в любое состояние.
Из состояния <<Немного текста>> можно попасть в любое состояние, кроме <<Фона>>.
Поэтому на рисунке \ref{fig:fsm-slim} изображена редуцированная диаграмма конечного автомата, опускающая состояния <<Фон>> и <<Немного текста>>.

\begin{figure}[H]
	\centering
	\includegraphics[width=\textwidth]{diag/fsm.slim.pdf}
	\caption{Конечный автомат, состояния кроме <<Немного текста>> и <<Фон>>}
	\label{fig:fsm-slim}
\end{figure}

На рисунке \ref{fig:primary-markup} ниже изображена схема алгоритма классификации конкретной сканирующей строки.

\begin{figure}[H]
	\centering
	\includegraphics[width=0.7\textwidth]{diag/primary-markup.pdf}
	\caption{Разметка сканирующей строки}
	\label{fig:primary-markup}
\end{figure}

\subsubsection*{Классификация строки}

Классификация строки производится на основе следующих параметров распределения пикселей в ней:
\begin{enumerate}
    \item count\_white --- количество белых пикселей в строке;
    \item count\_color --- количество цветных пикселей в строке;
    \item count\_gray --- количество серых (почти черных) пикселей в строке;
    \item comp\_lengths --- массив длин участков подряд идущих не белых пикселей;
    \item gap\_lengths --- массив длин промежутков (белых пикселей) между участками подряд идущих не белых пикселей;
    \item gray\_comp\_lengths --- массив длин участков подряд идущих серых пикселей;
    \item color\_comp\_lengths --- массив длин участков подряд идущих цветных пикселей;
    \item first\_nonwhite\_index --- индекс первого не белого пикселя в строке.
\end{enumerate}

\subsubsection*{Классификация как <<Фон>>}

Строка классифицируется как <<Фон>>, если выполнено условие <<first-\_nonwhite\_index не установлен>> --- в строке не нашлось не белых пикселей.

\subsubsection*{Классификация как <<Длинная черная линия>>}

Строка классифицируется как <<Длинная черная линия>>, если: строка содержит единственную серую компоненту И эта компонента достаточно длинная.

Содержание единственной серой компоненты определяется на основе одновременного выполнения следующих условий:
\begin{itemize}
    \item Длина comp\_lengths равна 1;
    \item Длина gap\_lengths равна 0;
    \item Длина color\_comp\_lengths равна 0.
\end{itemize}

Длина серой компоненты для классификации строки как <<Длинная черная линия>> считается достаточно большой, если count\_gray больше некоторого параметра, который является произведением длины строки на некоторую наперед заданную константу, принимающую значения от 0 до 1, например 1/2.

\subsubsection*{Классификация как <<Черная линия средней длины>>}

Строка классифицируется как <<Черная линия средней длины>>, если строка содержит компоненту, длина которой больше некоторого параметра, который является произведением длины строки на некоторую наперед заданную константу, принимающую значения от 0 до 1, например, 1/16.

\subsubsection*{Классификация как <<Много текста>>}

Строка классифицируется как <<Много текста>>, если она либо содержит очень много черных компонент, либо содержит много черных компонент и не содержит цвета.

Считается, что строка содержит очень много черных компонент, если длина comp\_lengths превышает некоторую наперед заданную константу, например 100.

Считается, что строка содержит много черных компонент, если длина comp\_lengths превышает некоторую наперед заданную константу, например 80.

Таким образом, если строка содержит очень много черных компонент, она будет классифицирована как <<Много текста>> вне зависимости от наличия в ней цвета.

\subsubsection*{Классификация как <<Цвет>>}

Строка классифицируется как <<Много текста>>, если count\_color больше нуля.

\subsubsection*{Классификация как <<Немного текста>>}

Строка классифицируется как <<Немного текста>>, если содержит немного компонент И компоненты преимущественно небольшого размера И промежутки между компонентами преимущественно небольшого размера И отсутствуют большие промежутки.

Считается, что компоненты и/или промеждутки между компонентами преимущественно небольшого размера, если среднее арифметическое comp-\_lengths и/или gap\_lengths меньше некоторой наперед заданной константы, например, 20 пикселей.

Считается, что большие промежутки отсутствуют, если в массиве gap\_lengths отсутствуют элементы с индексом стандартного отклонения больше шести.

Индекс стандартного отклонения $Z$ для элемента массива $x$ вычисляется по формуле:
$$
Z(x) = \frac{x - \mu}{\sigma},
$$
где $\mu$ --- среднее значение элементов массива, $\sigma$ --- стандартное отклонение элементов массива.

\subsubsection*{Классификация как <<Не определено>>}

В остальных случаях строка классифицируется как <<Не определено>>.

\subsubsection*{Изменение состояний КА}

\subsubsection{Уточненная разметка}

Исходные данные --- разметка страницы и информация о каждом сегменте на ней.
Получаемый результат --- уточненная разметка страницы.

Уточненная разметка страницы --- массив кортежей типа
$$
(y\_start, y\_end, C),
$$
где $y\_start$ --- $y$-координата начала сегмента в пространстве изображения, $y\_end$ --- $y$-координата конца сегмента в пространстве изображения, $C$ --- класс сегмента, где $C \subseteq$ $\{$Фон, Текст, Таблица, Листинг, Схема алгоритма, Рисунок, График, Не определено$\}$, причем координаты начала и конца сегментов совпадают с соответствующими координатами сегментов первичной разметки, уточняется только класс на основе информации о сегментах.

\subsubsection{Объединенная разметка}

Исходные данные --- уточненная разметка страницы.
Получаемый результат --- объединенная разметка страницы.

% При этом для каждого этапа следует выделить необходимые исходные данные и получаемые результаты

% Сформулировать и описать ключевые шаги метода в виде схем алгоритмов

% Разработать алгоритм, реализующий данный метод

\subsection{Тестирование и классы эквивалентности}

\subsection{Структура разрабатываемого ПО}

\subsection*{Вывод}
