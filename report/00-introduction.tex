\phantomsection\section*{ВВЕДЕНИЕ}\addcontentsline{toc}{section}{ВВЕДЕНИЕ}

% NOTE:
% Обоснование актуальности темы со ссылками на научные статьи
% Формулируется цель работы: "Целью данной работы является..."
% Перечисляются задачи для ее достижения (анал, конст, техно, иссл):
% "Для достижения поставленной цели необходимо решить следующие задачи"

Современная научная деятельность характеризуется стремительным ростом объёма публикуемых материалов.
По данным Clarivate~\cite{clara}, в 2023 году в базе Journal Citation Reports (JCR) насчитывалось более 21 500 научных журналов, охватывающих 254 научные дисциплины и представляющих 112 стран.
Такой масштаб публикационной активности требует эффективных методов обработки и анализа научных текстов.

Научно тексты обладают определенной структурой, включающей текст, таблицы, графики, схемы, рисунки и листинги.
Корректное выделение и идентификация этих элементов необходимы для таких задач, как автоматическое создание аннотаций, индексация информации, извлечение ключевых данных и повышение качества работы поисковых систем~\cite{cui, shiz}.

Несмотря на существование различных методов анализа структуры документов, большинство из них либо основаны на ресурсоёмких подходах с использованием глубокого машинного обучения, либо не позволяют эффективно выделять все специфические компоненты научно-технических текстов.
Это подтверждается сравнительным анализом методов, проведенным в аналитической части настоящей работы.
Например, традиционные методы, такие как анализ проекционного профиля и связных компонент, ограничены определёнными типами макетов и чувствительны к шумам и искажениям, в то время как методы на основе нейросетей требуют значительных вычислительных ресурсов и объемных размеченных данных для обучения~\cite{dla-book, dla-survey}.

Таким образом, актуальной задачей является разработка метода, который, опираясь на простые эвристические правила, позволял бы эффективно сегментировать и классифицировать компоненты научно-технических текстов без использования нейронных сетей.
Это позволит существенно сократить затраты вычислительных ресурсов и повысить скорость обработки документов.
На решение данной проблемы направлена настоящая дипломная работа.

Целью данной работы является разработка метода выделения составных частей научного текста на основе анализа распределения пикселей в сканирующей строке.

\newpage

Для достижения поставленной цели необходимо решить следующие задачи:
\begin{enumerate}
    \item рассмотреть и сравнить известные методы, которые могут быть применены для выделения составных частей научного текста;
    \item формализовать постановку задачи разработки метода выделения составных частей научного текста на основе анализа распределения пикселей в сканирующей строке;
    \item разработать описанный метод;
    \item разработать программное обеспечение, реализующее данный метод;
    \item провести исследование скорости разметки и максимального объема памяти, используемой в процессе разметки, в зависимости от количества процессов, участвующих в разметке.
\end{enumerate}
